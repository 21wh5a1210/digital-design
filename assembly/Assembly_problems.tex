
\begin{enumerate}

\item
\label{prob:gate IN 45}
 A portion of an assembly language program written for an $8$-bit microprocessor is given below along with explanations. The code is intended to introduce a software time delay. The processor is driven by a $5$ MHz clock. The time delay (in $\mu$s) introduced by the program is \vspace{12 pt}


MVI B, $64$H; Move immediate the given byte into register B. Takes $7$ clock periods.\vspace{12 pt}


LOOP: DCR B ; Decrement register B. Affects Flags. Takes $4$ clock periods. \vspace{12 pt}


JNZ LOOP ; Jump to address with Label LOOP if zero flag is not set. Takes $10$ clock periods when jump is performed and $7$ clock periods when jump is not performed.

\hfill(GATE IN 2018)

\item A $10\frac{1}{2}$ digit Counter-timer is set in the 'frequency mode' of operation (with $T_s=1s$). For a specific input, the reading obtained is $1000$. Without disconnecting this input, the Counter-timer is changed to operate in the 'Period mode' and the range selected is microseconds ($\mu$s,with $f_s=\SI{1}{\mega\hertz}$).The Counter Will then display
\hfill(GATE IN 2021)
\begin{enumerate}
  \item $0$
  \item $10$
  \item $100$
  \item $1000$
\end{enumerate}

\item Consider three registers $\textbf{R1 ,R2 ,R3}$ that store numbers in $ IEEE-754 $ single precision floating point format. Assume that $\textbf{R1}$ and $\textbf{R2}$ contain the values (in hexadecimal notation) $0\textbf{x}42200000$ and $0\textbf{x}C1200000,$ respectively. If $ \textbf{R3}=\frac{\textbf{R1}}{\textbf{R2}} $, what is the value stored in $\textbf{R3}?$
\hfill(GATE-EC2021,31)
\begin{enumerate}
    \item 0x40800000
    \item 0xC0800000
    \item 0x83400000
    \item 0xC8500000
\end{enumerate}

\end{enumerate}
